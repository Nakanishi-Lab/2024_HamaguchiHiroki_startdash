\documentclass[twocolumn]{jarticle}
\usepackage[dvipdfmx]{graphicx}
\usepackage[margin=20truemm]{geometry}
\setlength{\columnsep}{2zw}
\begin{document}

日本三景は以下の 3 つの名勝地を指す(記載順は全\\
国地方公共団体コードの順番による).全て海(沿岸)
にある風景となっており、各々古くから詩歌に詠まれ、
絵画に描かれていた.松島 - 宮城県宮城郡松島町を中
心とした多島海(地図)天橋立 - 京都府宮津市にある
砂嘴(地図)宮島(厳島) - 広島県廿日市市にある厳
島神社を中心とした島(地図)江戸時代前期の儒学者・
林春斎が、寛永 20 年 8 月 13 日(グレゴリオ暦 1643
年 9 月 25 日)に執筆した著書『日本国事跡考』の陸
奥国のくだりにおいて、「松島、此島之外有小島若干、
殆如盆池月波之景、境致之佳、與丹後天橋立・安藝嚴
嶋爲三處奇觀」(句読点等は筆者付記)と書き記した.
これを端緒に「日本三景」という括りが始まったとさ
れる.\\
\begin{figure}[htbp]
\begin{center}
\includegraphics[width=0.5\linewidth]{Desert.jpg}
\caption{ Desert}
\end{center}
\end{figure}\\
 その後、元禄 2 年閏 1 月 28 日(グレゴリオ暦 1689
年 3 月 19 日)に天橋立を訪れた儒学者・貝原益軒が、
その著書『己巳紀行』(きしきこう)の中の丹波丹後若
狭紀行において、天橋立を「日本の三景の一とするも
宜也」と記している.これが「日本三景」という言葉
の文献上の初出とされ、益軒が訪れる以前から「日本
三景」が一般に知られた括りであったと推定されてい
る.日本三景を雪月花にあてる場合、「雪」は天橋立、
「月」は松島、「花」は紅葉を花に見立てて宮島をあて
ている.\\
 近年の年間観光客数は、松島が約 370 万人(奥松島
や塩竈などを含む松島全体では 622 万人)、宮島が約
309 万人(対岸も含めた廿日市市全体では 562 万人)、
天橋立が約 267 万人(阿蘇海に面する宮津市と与謝野
町の合計は約 371 万人)となっている.観光大使とし
て、松島には「松島キャンペーンレディ」、天橋立に
は「プリンセス天橋立」、宮島には「宮島観光親善大
使」がおり、日本三景共同キャンペーンの際などに一
\begin{figure}[htbp]
\begin{center}
\includegraphics[width=0.5\linewidth]{Tulips.jpg}
\caption{ Tulips}
\end{center}
\end{figure}\\
緒に活動している.夏には、松島灯籠流し花火大会、
宮津灯籠流し花火大会、宮島水中花火大会という海上
花火が日本三景各地で開催され、多くの観光客を集め
る.いずれも名物として牡蠣があるが、松島と宮島は
冬が主なのに対し、天橋立は夏の岩ガキが主である.
松島の牡蠣鍋クルーズや寿司(特定第 3 種漁港・塩釜
漁港)、天橋立の松葉ガニやとり貝、宮島のもみじ饅
頭やアナゴが訴求力のある食観光として知られる.
日本三景はいずれも 1952 年(昭和 27 年)11 月 22 日
に特別名勝に指定されている.日本三景は各々別々に
世界遺産登録に動いたが、現時点では厳島神社(1996
年(平成 8 年)12 月登録)以外は登録に至っていない.
2006 年(平成 18 年)7 月 5 日に開催された日本三景観
光連絡協議会の総会において、林春斎の誕生日が元和
4 年 5 月 29 日(グレゴリオ暦 1618 年 7 月 21 日)であ
ることに因み、7 月 21 日を「日本三景の日」と制定し
た.2007 年(平成 19 年)4 月、日本を訪れる外国人観
光客向けに、ミシュラン実用旅行ガイド (MICHELIN
Voyager Pratique Japon) が発刊された.これは「ビ
ジット・ジャパン・キャンペーン」の一環として、ミ
シュラン社や国土交通省などが連携したものである.日
本三景では松島と宮島が単独の項目として記載され、
各々三ツ星が 3 つずつ付いた.
\begin{figure}[htbp]
    \begin{center}
    \includegraphics[width=0.5\linewidth]{koala.jpg}
    \caption{ koala}
    \end{center}
    \end{figure}\\
\end{document}